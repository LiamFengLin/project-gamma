\documentclass[11pt]{article}

\usepackage[margin=0.75in]{geometry}

\title{Analysis of fcMRI data in Schizophrenia}
\author{
  Hejazi, Nima\\
  \texttt{nhejazi}
  \and
  Lin, Feng\\
  \texttt{LiamFengLin}
  \and
 Zhao, Luyun\\
  \texttt{lynnzhao}
  \and
  Zhou, Xinyue\\
  \texttt{z357412526}
}

\bibliographystyle{siam}

\begin{document}
\maketitle

\abstract{You should have a short abstract.}

\section{Introduction}

Identify a published fMRI paper and the accompanying data
\cite{lindquist2008statistical}.  You should explain the basic idea of the
paper in a paragraph.  You should also perform basic sanity check on the data
(e.g., can you downloaded, can you load the files, confirm that you have the
correct number of subjects).

Briefly explain what reproducibility means and in what sense you will
try to reproduce this study.

\section{Data}

\section{Methods}
\section{Results}
\section{Discussion}

\section{Proposal}

For this project, we seek to understand and analyze the data from the fMRI 
studies conducted by Repovs \textit{et al.} on the manner in which brain 
network connectivity is related to schizophrenia \cite{repovs2011,repovs2012}.
As reported, Repovs \textit{et al.} used a number of brain regions, selected
\textit{a priori}, to examine the manner in which connectivity, as measured 
by correlation in voxel activity across regions, differed across and between 
controls and individuals with schizophrenia. The cited papers examine the 
alterations in functional connectivity within brain networks in order to 
provide evidence that schizophrenia partially reflects a disconnection 
syndrome. Specifically, resting state fMRI data was collected on 40 
individuals with schizophrenia as well as 31 siblings of individuals with 
schizophrenia, as well as 15 healthy controls and 18 siblings of controls
\cite{repovs2011}. With data from the experiment, the paper concludes that 
individuals with schizophrenia showed reduced distal and enhanced local 
connectivity between the cognitive control networks. Moreover, greater 
connectivity between the frontal-parietal and cerebellar regions was highly
correlated with better cognitive performance across groups. In performing a few
basic sanity checks, we found that we were successful in downloading the data 
and properly loading it into R and Python to perform basic data manipulations,
including verification that we did have the right number of subjects and that 
labeling was correct within the data set as provided.

As a first step in our analysis, we wish to reproduce the basic patterns and
assumptions noted in the paper (e.g. we want to identify brain regions that
correspond to the Default Mode Network (DMN) and the differences between
healthy and schizophrenic individuals at resting state). At this early step,
we will leverage only summary statistics (mean, variance) averaged across 
regions of the brain at arbitrary time steps. We should be able to detect 
differences in fMRI data between the two groups of individuals using this 
naive approach. We will likely need to perform Talairach coordinate system 
transformation and potentially motion correction in order to perform valid 
inference. Using domain-specific knowledge from the literature to guide us, 
we hope to devise techniques for detecting known patterns in the fMRI data. 
Since this is an analytic study in statistics, we will perform preprocessing
steps on the data only with proper justification as to their effects on the 
analysis results. As an example, to compensate for slice-dependent time shifts,
we will simply remove the first five images from each run during which BOLD 
signal was allowed to reach steady state. In addition, we may need to perform
Talairach coordinate transformation and motion correction if necessary when 
we are further in the analytic process. To reproduce the results noted in 
\cite{repovs2011}, we will compute the average connectivity, through 
correlations and the Fisher r-to-z transformation; we will then apply separate
repeated measures of ANOVAs to compare the groups within and between network
connectivity. In order to ensure that the analyses provide valid inference, 
the False Discovery Rate will be applied to control for multiple comparisons. 
As a final step in our analysis, we hope to use simple machine learning methods
to predict whether a subject is schizophrenic by training a classifier using
cross-validation to achieve enhanced predictive value \cite{arbabshirani2013}.

\bibliography{project}

\end{document}
